\documentclass[10pt,a4paper]{article}
\usepackage[utf8]{inputenc}
\usepackage[english]{babel}
\usepackage{amsmath}
\usepackage{amsfonts}
\usepackage{amssymb}
\usepackage{graphicx}
\usepackage[left=2cm,right=2cm,top=2cm,bottom=2cm]{geometry}
\author{Arun Prasaad Gunasekaran}
\title{Equations}
\begin{document}
\maketitle

\begin{eqnarray}
x^2 + 2x + 1 &=& (x+1)^2\\
%
x^3 + 3x^2 + 3x + 1 &=& \nonumber \\
(x+1)^3
% Use & for manual alignment
\end{eqnarray}

\begin{eqnarray*}
x^2 + 2x + 1 = (x+1)^2\\
x^3 + 3x^2 + 3x +1 = \\
\left[(x+1)^3\right]\\
\left[-(x+1)^3\right]\\
\left[(x+1)^3\right]\\
\end{eqnarray*}

\begin{align}
x^2 + 2x + 1 = (x+1)^2  \\
x^3 + 3x^2 + 3x +1 = \nonumber \\
(x+1)^3\\
\intertext{This is a set of identities in algebra}
\end{align}

\begin{align*}
x^2 + 2x + 1 &=& (x+1)^2\\
x^3 + 3x^2 + 3x +1 &=& \\
(x+1)^3
\intertext{This is a set of identities in algebra}
\end{align*}

\begin{align}
\label{iden1}
x^2 + 2x + 1 = (x+1)^2\\
\label{iden2}
x^3 + 3x^2 + 3x +1 = \nonumber\\
\left[(x+1)^3
-(x+1)^3
(x+1)^3\right]\intertext{This is a set of identities in algebra}
\end{align}

Equations \ref{iden1} and \ref{iden2} expands identities

\end{document}
