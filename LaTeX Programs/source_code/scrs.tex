\documentclass[10pt,a4paper]{article}
\usepackage[utf8]{inputenc}
\usepackage[english]{babel}
\usepackage{amsmath}
\usepackage{amsfonts}
\usepackage{amssymb}

\usepackage{listings} % Necessary For using Listing commands

\usepackage{color} % For making your own colours

\definecolor{mygreen}{rgb}{0,0.6,0} % R G B values
\definecolor{mygray}{rgb}{0.5,0.5,0.5}
\definecolor{mymauve}{rgb}{0.58,0,0.82}

\usepackage[left=2cm,right=2cm,top=2cm,bottom=2cm]{geometry}
\author{Arun Prasaad Gunasekaran}
\title{Source Codes}

\lstset{
 backgroundcolor=\color{white},   % choose the background color; you must add \usepackage{color} or \usepackage{xcolor}
 basicstyle=\footnotesize,        % the size of the fonts that are used for the code
 breakatwhitespace=false,         % sets if automatic breaks should only happen at whitespace
 breaklines=true,                 % sets automatic line breaking
 captionpos=b,                    % sets the caption-position to bottom
 commentstyle=\color{mygreen},    % comment style
 deletekeywords={...},            % if you want to delete keywords from the given language
 escapeinside={\%*}{*)},          % if you want to add LaTeX within your code
 extendedchars=true,              % lets you use non-ASCII characters; for 8-bits encodings only, does not work with UTF-8
 frame=single,	                   % adds a frame around the code
 keepspaces=true,                 % keeps spaces in text, useful for keeping indentation of code (possibly needs columns=flexible)
  keywordstyle=\color{blue},       % keyword style
  language=Python,                 % the language of the code
  otherkeywords={*,...},            % if you want to add more keywords to the set
  numbers=left,                    % where to put the line-numbers; possible values are (none, left, right)
  numbersep=5pt,                   % how far the line-numbers are from the code
  numberstyle=\tiny\color{mygray}, % the style that is used for the line-numbers
  rulecolor=\color{red},         % if not set, the frame-color may be changed on line-breaks within not-black text (e.g. comments (green here))
  showspaces=false,                % show spaces everywhere adding particular underscores; it overrides 'showstringspaces'
  showstringspaces=false,          % underline spaces within strings only
  showtabs=true,                  % show tabs within strings adding particular underscores
  stepnumber=2,                    % the step between two line-numbers. If it's 1, each line will be numbered
  stringstyle=\color{mymauve},     % string literal style
  tabsize=4,	                   % sets default tabsize to 4 spaces
  title=\lstname                   % show the filename of files included with \lstinputlisting; also try caption instead of title
}

\begin{document}
\maketitle
Hello!

\begin{verbatim}
# -*- coding: utf-8 -*-
"""
Created on Wed Sep  2 07:34:25 2015

@author: arun
"""

import numpy as np
from scipy.integrate import odeint
import matplotlib.pyplot as plt

def dampfunc(x, t):
    F0 = 50.0
    omg = 1.0
    m = 2.0
    c = 1.0
    k = 20
    x0 = x[0] # Displacement
    x1 = x[1] # Velocity
    x2 = F0*np.sin(omg*t)/m - c*x1/m - k*x0/m # Acceleration
    return x1, x2

F0 = 50.0
omg = 1.0
m = 2.0
c = 1.0
k = 20

init = [0.0, 0.0] # Initial conditions
t = np.linspace(0.0, 10.0, 101)

x = odeint(dampfunc, init, t) # This is the vital part!

mass = x

acc = F0*np.sin(omg*t)/m - c*x[:,1]/m - k*x[:,0]/m

plt.plot(t, x[:,0], 'b', label='$x (m)$')
plt.plot(t, x[:,1], 'r', label='$\dot{x} (m/s)$')
plt.plot(t, acc, 'g', label='$\ddot{x} (m/s^2)$')
plt.axis([0, 10, -10, 10])
plt.xticks(np.linspace(0,10,11))
plt.yticks(np.linspace(-10,10,11))
plt.grid('on')
plt.legend(framealpha=0.5)
plt.title('Solution to a Mass Spring Damper System')
plt.suptitle('Session 3, Demos')
plt.xlabel('Time in s')
plt.ylabel('Magnitudes')
plt.savefig('Mass_Spring_Damper.jpg', format='jpg', dpi=200)
\end{verbatim}

\begin{lstlisting}[language=Python, frame=single]
print "Hello World!"
# -*- coding: utf-8 -*-
"""
Created on Wed Sep  2 07:34:25 2015

@author: arun
"""

import numpy as np
from scipy.integrate import odeint
import matplotlib.pyplot as plt

def dampfunc(x, t):
    F0 = 50.0
    omg = 1.0
    m = 2.0
    c = 1.0
    k = 20
    x0 = x[0] # Displacement
    x1 = x[1] # Velocity
    x2 = F0*np.sin(omg*t)/m - c*x1/m - k*x0/m # Acceleration
    return x1, x2

F0 = 50.0
omg = 1.0
m = 2.0
c = 1.0
k = 20

init = [0.0, 0.0] # Initial conditions
t = np.linspace(0.0, 10.0, 101)

x = odeint(dampfunc, init, t) # This is the vital part!

mass = x

acc = F0*np.sin(omg*t)/m - c*x[:,1]/m - k*x[:,0]/m

plt.plot(t, x[:,0], 'b', label='$x (m)$')
plt.plot(t, x[:,1], 'r', label='$\dot{x} (m/s)$')
plt.plot(t, acc, 'g', label='$\ddot{x} (m/s^2)$')
plt.axis([0, 10, -10, 10])
plt.xticks(np.linspace(0,10,11))
plt.yticks(np.linspace(-10,10,11))
plt.grid('on')
plt.legend(framealpha=0.5)
plt.title('Solution to a Mass Spring Damper System')
plt.suptitle('Session 3, Demos')
plt.xlabel('Time in s')
plt.ylabel('Magnitudes')
plt.savefig('Mass_Spring_Damper.jpg', format='jpg', dpi=200)
\end{lstlisting}

\lstinputlisting[language=Python, firstline=1, lastline=50, frame=single]{ode_solve1.py}

\end{document}